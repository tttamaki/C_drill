\documentclass[a4j]{jarticle}
\usepackage{multicol}
\raggedbottom
\addtolength{\textwidth}{4cm}
\addtolength{\textheight}{3.5cm}
\addtolength{\topmargin}{-2.5cm}
\addtolength{\evensidemargin}{-2cm}
\addtolength{\oddsidemargin}{-2cm}
\setlength{\columnseprule}{.5pt}
\setlength{\columnsep}{4zw}

\makeatletter
\def\verbatim@font{\small\bfseries\ttfamily}
\makeatother


%
% 1か2を指定する。
%
\def\anss{1} % 1 : Question mode 学生に解かせるとき
             % 2 : Answer mode   回答例



\ifnum \anss=1
\def\an#1{\phantom{#1}} % Question mode
\else
\def\an#1{{#1}}  % Answer mode
\fi



\def\ds{\displaystyle}

\begin{document}
\thispagestyle{empty}

\begin{multicols*}{2}%


\def\subst#1#2{$\ds #1$
 \ $\longrightarrow$\ 
 \underline{\hbox to 5cm{\ttfamily #2}}}



\noindent
\begin{tabular}[t]{|c|cccccccc|}\hline
氏  名 & & & & & & & & \\ \hline
\end{tabular}\\
\begin{tabular}[t]{|c|c|c|c|c|c|c|c|c|c|}\hline
学籍番号 & & & & & & & & \\ \hline
\end{tabular}\\
学籍番号の\underline{数字の}右から一番目が{\bfseries 奇数の人は左側}の問題を
解いて下さい。
\vspace{-5ex}


\subsection*{問1}



1. 符合無し整数(ただしlong)
の変数{\ttfamily xとyとz}の変数宣言を書け。\\
\an{{\ttfamily unsigned long int x, y, z;}}\\


2. {\ttfamily 3/10}の評価結果は  \an{0}  である。





\subsection*{問2}

次のプログラム終了時には何が表示されるか。
\begin{verbatim}
char b1, b2;
b1 = 'k' - 'K' + 'J';
b2 = 'm' - 4;
printf("%d %c\n", b1, b1);
printf("%d %c\n", b2, b2);
\end{verbatim}

\ifnum \anss=1
\vspace*{1.5cm}
\else
\begin{verbatim}
106 j
105 i
\end{verbatim}
\fi


\subsection*{問3}

次のプログラム終了時には何が表示されるか。
\begin{verbatim}
int a[3] = {2, 3, 1}, x = 1, b[3] = {5, 4, 8};
printf("%d %d %d\n", a[a[0]], a[1], a[2]);
printf("%d %d %d\n", b[a[0]-2], b[a[1]-3], b[a[2]-1]);
printf("%d %d %d\n", b[x + 1], b[x], b[x - 1]);
\end{verbatim}

\ifnum \anss=1
\vspace*{1.5cm}
\else
\begin{verbatim}
1 3 1
5 5 5
8 4 5
\end{verbatim}
\fi



\subsection*{問4}

次のプログラム終了時には何が表示されるか。
\begin{verbatim}
int i = 0;
printf("%d ", i);
i = i + 1;
printf("%d\n", i);
i = i + 1;
printf("%d ", i);
i = i + 1;
printf("end\n");
\end{verbatim}

\ifnum \anss=1
\vspace{1.5cm}
\else
\begin{verbatim}
0 1
2 end
\end{verbatim}\vspace{2cm}
\fi


\vfill

\mbox{}

%{\vbox{\vspace{1cm}}}




\noindent
\begin{tabular}[t]{|c|cccccccc|}\hline
氏  名 & & & & & & & & \\ \hline
\end{tabular}\\
\begin{tabular}[t]{|c|c|c|c|c|c|c|c|c|c|}\hline
学籍番号 & & & & & & & & \\ \hline
\end{tabular}\\
学籍番号の\underline{数字の}右から一番目が{\bfseries 偶数の人は右側}の問題を
解いて下さい。
\vspace{-5ex}





\subsection*{問1}

1. 実数型の変数{\ttfamily a}と{\ttfamily b}の変数宣言を書け。\\
\an{{\ttfamily double a, b;} または {\ttfamily float a, b;}}\\

2. {\ttfamily 10/3}の評価結果は  \an{3}  である。





\subsection*{問2}

次のプログラム終了時には何が表示されるか。
\begin{verbatim}
char b1, b2;
b1 = 'j' - ('c' - 'C');
b2 = 'I' - 7;
printf("%d %c\n", b1, b1);
printf("%d %c\n", b2, b2);
\end{verbatim}

\ifnum \anss=1
\vspace*{2cm}
\else
\begin{verbatim}
74 J
66 B
\end{verbatim}
\fi


\subsection*{問3}

次のプログラム終了時には何が表示されるか。
\begin{verbatim}
int a[3] = {2, 3, 1}, x = 2, b[3] = {5, 4, 8};
printf("%d %d %d\n", a[0], a[1], a[a[2]]);
printf("%d %d %d\n", b[a[0]-1], b[a[1]-2], b[a[2]]);
printf("%d %d %d\n", b[x - 2] + 3, b[x - 1] + 4, b[x]);
\end{verbatim}

\ifnum \anss=1
\vspace*{2cm}
\else
\begin{verbatim}
2 3 3
4 4 4
8 8 8
\end{verbatim}
\fi



\subsection*{問4}

次のプログラム終了時には何が表示されるか。
\begin{verbatim}
int i = -2;
printf("%d\n ", i);
i = i + 2;
printf("%d ", i);
i = i + 2;
printf("%d ", i);
i = i + 2;
printf("end\n");
\end{verbatim}


\ifnum \anss=1
\vspace{2cm}
\else
\begin{verbatim}
-2
0 2 end
\end{verbatim}
\fi



%\vfill




\end{multicols*}


\end{document}
