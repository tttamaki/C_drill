\documentclass[a4j]{jarticle}
\usepackage{multicol}
\raggedbottom
\addtolength{\textwidth}{4cm}
\addtolength{\textheight}{3.5cm}
\addtolength{\topmargin}{-2.5cm}
\addtolength{\evensidemargin}{-2cm}
\addtolength{\oddsidemargin}{-2cm}
\setlength{\columnseprule}{.5pt}
\setlength{\columnsep}{4zw}

\makeatletter
\def\verbatim@font{\small\bfseries\ttfamily}
\makeatother


%
% 1か2を指定する。
%
\def\anss{1} % 1 : Question mode 学生に解かせるとき
             % 2 : Answer mode   回答例



\ifnum \anss=1
\def\an#1{\phantom{#1}} % Question mode
\else
\def\an#1{{#1}}  % Answer mode
\fi



\def\ans#1#2{
\ifnum \anss=1
#1
\else
#2
\fi
}




\def\ds{\displaystyle}

\begin{document}
\thispagestyle{empty}

\begin{multicols*}{2}%


\def\subst#1#2{$\ds #1$
 \ $\longrightarrow$\ 
 \underline{\hbox to 5cm{\ttfamily #2}}}



\noindent
\begin{tabular}[t]{|c|cccccccc|}\hline
氏  名 & & & & & & & & \\ \hline
\end{tabular}\\
\begin{tabular}[t]{|c|c|c|c|c|c|c|c|c|c|}\hline
学籍番号 & & & & & & & & \\ \hline
\end{tabular}\\
学籍番号の\underline{数字の}右から一番目が{\bfseries 奇数の人は左側}の問題を
解いて下さい。
\vspace{-5ex}







\subsection*{問1}

以下の配列{\ttfamily a}の要素を全て表示したい。
{\ttfamily while}を使って、そのためのプログラムの続きを書け。
変数{\ttfamily i}を使ってよい。


\ifnum \anss=1
\begin{verbatim}
int i;
double a[4] = {7.7, 3.9, 1.1, 5.6};







\end{verbatim}
\else
\begin{verbatim}
int i;
double a[4] = {7.7, 3.9, 1.1, 5.6};
i = 0;
while(i < 4){
  printf("%f ", a[i]);
  i = i + 1;
}
\end{verbatim}
\fi
\vspace{5cm}



\subsection*{問2}

二重ループを使って
$1\times1$から$9\times9$までのかけ算の結果の表(九九)を
表示するプログラムを完成させよ。


\ifnum \anss=1
\begin{verbatim}



  while(i <= 9){



    while(j <= 9){




    }




  }
\end{verbatim}
\vspace{1cm}
\else
\begin{verbatim}
  int i = 1, j;
  while(i <= 9){
    j = 1;
    while(j <= 9){
      printf("%d ", i*j);
      j = j + 1;
    }
    i = i + 1;
    printf("\n");
  }
\end{verbatim}
\vspace{5cm}
\fi




\vfill

\mbox{}


%
%
%
%\subsection*{問1}
%
%{\ttfamily while}文を用いたプログラムに書き直せ。
%\begin{verbatim}
%int r = -4;
%if(r > -3){
%  printf("%d ", r);
%}
%r = r + 2;
%if(r > -3){
%  printf("%d ", r);
%}
%r = r + 2;
%if(r > -3){
%  printf("%d ", r);
%}
%r = r + 2;
%printf("\n");
%\end{verbatim}
%
%
%\noindent
%解答欄:
%%\begin{verbatim}
%%int r = -4;
%%while(r < 1){
%%  if(r > -3){
%%    printf("%d ", r);
%%  }
%%  r = r + 2;
%%}
%%printf("\n");
%%\end{verbatim}\vspace*{2cm}
%\begin{verbatim}
%
%
%
%
%
%
%
%
%\end{verbatim}\vspace*{2cm}
%
%
%
%\subsection*{問2}
%
%
%次のプログラムがある。
%\begin{verbatim}
%int i = 0, a[6] = {11, 56, 34, 77, 39, 90};
%while(i < 6){
%  if(a[i] < 50){
%     printf("## ");
%  }else{
%     printf( _____________ , a[i]); 
%  }
%  __________
%}
%\end{verbatim}
%これを実行したときに、以下のように表示したい。
%\begin{verbatim}
%## P56 is OK. ## P77 is OK. ## P90 is OK.
%\end{verbatim}
%プログラム中の下線部を埋めよ。
%
%
%\vspace*{1cm}
%
%
%\vfill
%
%\mbox{}
%
%%{\vbox{\vspace{1cm}}}
%
%




\noindent
\begin{tabular}[t]{|c|cccccccc|}\hline
氏  名 & & & & & & & & \\ \hline
\end{tabular}\\
\begin{tabular}[t]{|c|c|c|c|c|c|c|c|c|c|}\hline
学籍番号 & & & & & & & & \\ \hline
\end{tabular}\\
学籍番号の\underline{数字の}右から一番目が{\bfseries 偶数の人は右側}の問題を
解いて下さい。
\vspace{-5ex}







\subsection*{問1}

以下の配列{\ttfamily b}の要素を全て表示したい。
{\ttfamily while}を使って、そのためのプログラムの続きを書け。
変数{\ttfamily j}を使ってよい。


\ifnum \anss=1
\begin{verbatim}
int j;
float b[6] = {0.1, 0.5, 0.3, 0.7, 0.9, 0.2};







\end{verbatim}
\else
\begin{verbatim}
int j;
float b[6] = {0.1, 0.5, 0.3, 0.7, 0.9, 0.2};
j = 0;
while(j < 6){
  printf("%f ", b[j]);
  j = j + 1;
}
\end{verbatim}
\fi
\vspace{5cm}



\subsection*{問2}



二重ループを使って
$1\times1$から$9\times9$までのかけ算の結果の表(九九)を
表示するプログラムを完成させよ。


\ifnum \anss=1
\begin{verbatim}



  while(a <= 9){



    while(b <= 9){




    }




  }
\end{verbatim}
\vspace{1cm}
\else
\begin{verbatim}
  int a = 1, b;
  while(a <= 9){
    b = 1;
    while(b <= 9){
      printf("%d ", a*b);
      b = b + 1;
    }
    a = a + 1;
    printf("\n");
  }
\end{verbatim}
\vspace{5cm}
\fi







%
%
%\subsection*{問1}
%
%
%{\ttfamily while}文を用いたプログラムに書き直せ。
%\begin{verbatim}
%int w = 9;
%if(w < 7){
%  printf("%d ", w);
%}
%w = w - 2;
%if(w < 7){
%  printf("%d ", w);
%}
%w = w - 2;
%if(w < 7){
%  printf("%d ", w);
%}
%w = w - 2;
%printf("\n");
%\end{verbatim}
%
%
%\noindent
%解答欄:
%%\begin{verbatim}
%%int w = 9;
%%while(w > 4){
%%  if(w < 7){
%%    printf("%d ", w);
%%  }
%%  w = w - 2;
%%}
%%printf("\n");
%%\end{verbatim}\vspace*{2cm}
%\begin{verbatim}
%
%
%
%
%
%
%
%
%\end{verbatim}\vspace*{2cm}
%
%
%
%
%\subsection*{問2}
%
%
%次のプログラムがある。
%\begin{verbatim}
%int i = 0, a[6] = {11, 56, 34, 77, 39, 90};
%while(i < 6){
%  if(a[i] > 40){
%     printf( _____________ , a[i]); 
%  }else{
%     printf("$$ ");
%  }
%  __________
%}
%\end{verbatim}
%これを実行したときに、以下のように表示したい。
%\begin{verbatim}
%$$ Test 56th $$ Test 77th $$ Test 90th
%\end{verbatim}
%プログラム中の下線部を埋めよ。
%
%
%
%%\vfill
%



\end{multicols*}


\end{document}
