\documentclass[a4j]{jarticle}
\usepackage{multicol}
\raggedbottom
\addtolength{\textwidth}{4cm}
\addtolength{\textheight}{3.5cm}
\addtolength{\topmargin}{-2.5cm}
\addtolength{\evensidemargin}{-2cm}
\addtolength{\oddsidemargin}{-2cm}
\setlength{\columnseprule}{.5pt}
\setlength{\columnsep}{4zw}

\makeatletter
\def\verbatim@font{\small\bfseries\ttfamily}
\makeatother


%
% 1か2を指定する。
%
\def\anss{1} % 1 : Question mode 学生に解かせるとき
             % 2 : Answer mode   回答例



\ifnum \anss=1
\def\an#1{\phantom{#1}} % Question mode
\else
\def\an#1{{#1}}  % Answer mode
\fi



\def\ds{\displaystyle}

\begin{document}
\thispagestyle{empty}

\begin{multicols*}{2}%


\def\subst#1#2{$\ds #1$
 \ $\longrightarrow$\ 
 \underline{\hbox to 5cm{\ttfamily #2}}}



\noindent
\begin{tabular}[t]{|c|cccccccc|}\hline
氏  名 & & & & & & & & \\ \hline
\end{tabular}\\
\begin{tabular}[t]{|c|c|c|c|c|c|c|c|c|c|}\hline
学籍番号 & & & & & & & & \\ \hline
\end{tabular}\\
学籍番号の\underline{数字の}右から一番目が{\bfseries 奇数の人は左側}の問題を
解いて下さい。
\vspace{-5ex}


\subsection*{問1}
%
%1. 実数型({\ttfamily double})の変数{\ttfamily Sigma\_1}と
%符号無し整数({\ttfamily long})の変数
%{\ttfamily count0}の変数宣言を書け。\\
%\an{{\ttfamily double Sigma\_1;}}\\
%\an{{\ttfamily unsigned long int count0;}}\\
%
%
%2. {\ttfamily 5/2}の評価結果は  \an{2}  である。




1. 実数型({\ttfamily float})の変数{\ttfamily Sigma\_0}と
符号付き整数({\ttfamily short})の変数
{\ttfamily count1}の変数宣言を書け。\\
\an{{\ttfamily float Sigma\_0;}}\\
\an{{\ttfamily signed short int count1;}}\\


2. {\ttfamily 5/2}の評価結果は  \an{2}  である。





\subsection*{問2}

次のプログラム終了時には何が表示されるか。
\begin{verbatim}
int a[3] = {1, 2, 3};
char b[3] = {'e', 'f', 'g'};
b[0] = b[0] + a[0];
b[1] = b[1] + a[1];
b[2] = b[2] + a[2];
printf("%c %c %c\n", b[0], b[1], b[2]);
\end{verbatim}

\ifnum \anss=1
\vspace*{1cm}
\else
\begin{verbatim}
f h j
\end{verbatim}
\fi







\subsection*{問3}

次のプログラム終了時には{\ttfamily a)}から{\ttfamily f)}のどれ
が表示されるか。\an{答:{\ttfamily c)}}
\begin{verbatim}
int i = -3;
printf("%d ", i);
i = i + 3;
printf("%d\n", i);
i = i + 3;
printf("%d ", i);
i = i + 3;
printf("\n");
\end{verbatim}

\begin{verbatim}
a)      -3 0 3

b)      -3 0 3 6

c)      -3 0
        3

d)      -3 0
        3 6

e)      -3
        0
        3
        6

f)      -3
        0
        3
\end{verbatim}




\vfill

\mbox{}

%{\vbox{\vspace{1cm}}}




\noindent
\begin{tabular}[t]{|c|cccccccc|}\hline
氏  名 & & & & & & & & \\ \hline
\end{tabular}\\
\begin{tabular}[t]{|c|c|c|c|c|c|c|c|c|c|}\hline
学籍番号 & & & & & & & & \\ \hline
\end{tabular}\\
学籍番号の\underline{数字の}右から一番目が{\bfseries 偶数の人は右側}の問題を
解いて下さい。
\vspace{-5ex}





\subsection*{問1}

%1. 符号付き文字型の変数{\ttfamily Name}と
%実数型({\ttfamily float})の変数{\ttfamily Value}
%の変数宣言を書け。\\
%\an{{\ttfamily signed char Name;}}\\
%\an{{\ttfamily float Value;}}\\
%
%
%2. {\ttfamily 1/5}の評価結果は  \an{0}  である。



1. 実数型({\ttfamily double})の変数{\ttfamily Sigma\_1}と
符号無し整数({\ttfamily long})の変数
{\ttfamily count0}の変数宣言を書け。\\
\an{{\ttfamily double Sigma\_1;}}\\
\an{{\ttfamily unsigned long int count0;}}\\


2. {\ttfamily 2/3}の評価結果は  \an{0}  である。







\subsection*{問2}

次のプログラム終了時には何が表示されるか。
\begin{verbatim}
char a[3], b[3] = {'0', '1', '2'};
a[0] = b[0] + 1;
a[1] = b[1] + 1;
a[2] = b[2] + 1;
printf("%c %c %c\n", a[0]+1, a[1]+2, a[2]+3);
\end{verbatim}


\ifnum \anss=1
\vspace*{1cm}
\else
\begin{verbatim}
2 4 6
\end{verbatim}
\fi






\subsection*{問3}



次のプログラム終了時には{\ttfamily a)}から{\ttfamily f)}のどれ
が表示されるか。\an{答:{\ttfamily d)}}
\begin{verbatim}
int i = 9;
printf("%d\n", i);
i = i - 4;
printf("%d ", i);
i = i - 4;
printf("%d ", i);
i = i - 4;
printf("\n");
\end{verbatim}

\begin{verbatim}
a)      9 5 1

b)      9 5 1 -3

c)      9
        5 1 -3

d)      9
        5 1

e)      9
        5
        1
        -3

f)      9
        5
        1
\end{verbatim}



%\vfill




\end{multicols*}


\end{document}
