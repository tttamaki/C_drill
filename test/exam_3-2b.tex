\documentclass[a4j]{jarticle}
\usepackage{multicol}
\raggedbottom
\addtolength{\textwidth}{4cm}
\addtolength{\textheight}{3.5cm}
\addtolength{\topmargin}{-2.5cm}
\addtolength{\evensidemargin}{-2cm}
\addtolength{\oddsidemargin}{-2cm}
\setlength{\columnseprule}{.5pt}
\setlength{\columnsep}{4zw}

\makeatletter
\def\verbatim@font{\small\bfseries\ttfamily}
\makeatother


%
% 1か2を指定する。
%
\def\anss{1} % 1 : Question mode 学生に解かせるとき
             % 2 : Answer mode   回答例



\ifnum \anss=1
\def\an#1{\phantom{#1}} % Question mode
\else
\def\an#1{{#1}}  % Answer mode
\fi


\def\ans#1#2{
\ifnum \anss=1
#1
\else
#2
\fi
}




\def\ds{\displaystyle}

\begin{document}
\thispagestyle{empty}

\begin{multicols*}{2}%


\def\subst#1#2{$\ds #1$
 \ $\longrightarrow$\ 
 \underline{\hbox to 5cm{\ttfamily #2}}}



\noindent
\begin{tabular}[t]{|c|cccccccc|}\hline
氏  名 & & & & & & & & \\ \hline
\end{tabular}\\
\begin{tabular}[t]{|c|c|c|c|c|c|c|c|c|c|}\hline
学籍番号 & & & & & & & & \\ \hline
\end{tabular}\\
学籍番号の\underline{数字の}右から一番目が{\bfseries 奇数の人は左側}の問題を
解いて下さい。
\vspace{-5ex}






\subsection*{問1}

以下の配列{\ttfamily a}の要素を全て表示したい。
{\ttfamily while}を使って、そのためのプログラムの続きを書け。
変数{\ttfamily i}を使ってよい。


\ifnum \anss=1
\begin{verbatim}
int i;
float a[5] = {1.1, 5.6, 3.4, 7.7, 3.9};







\end{verbatim}
\else
\begin{verbatim}
int i;
float a[5] = {1.1, 5.6, 3.4, 7.7, 3.9};
i = 0;
while(i < 5){
  printf("%f ", a[i]);
  i = i + 1;
}
\end{verbatim}
\fi
\vspace{5cm}



\subsection*{問2}

二重ループを使って
$1\times1$から$9\times9$までのかけ算(九九)の表を
表示するプログラムを完成させよ。


\ifnum \anss=1
\begin{verbatim}



  while(i <= 9){



    while(j <= 9){




    }




  }
\end{verbatim}
\vspace{1cm}
\else
\begin{verbatim}
  int i = 1, j;
  while(i <= 9){
    j = 1;
    while(j <= 9){
      printf("%d ", i*j);
      j = j + 1;
    }
    i = i + 1;
    printf("\n");
  }
\end{verbatim}
\vspace{5cm}
\fi




%以下の配列{\ttfamily a}の、20以上の要素を全て表示したい。
%{\ttfamily while}を使って、そのためのプログラムを書け。
%変数{\ttfamily i}を使ってよい。
%
%
%\ifnum \anss=1
%\begin{verbatim}
%int i, a[5] = {11, 56, 34, 77, 39};
%
%
%
%
%
%
%
%\end{verbatim}
%\else
%\begin{verbatim}
%int i, a[5] = {11, 56, 34, 77, 39};
%i = 0;
%while(i < 5){
%  if(a[i] >= 20){
%    printf("%c ", a[i]);
%  }
%  i = i + 1;
%}
%\end{verbatim}
%\fi
%\vspace{5cm}







\vfill

\mbox{}

%{\vbox{\vspace{1cm}}}




\noindent
\begin{tabular}[t]{|c|cccccccc|}\hline
氏  名 & & & & & & & & \\ \hline
\end{tabular}\\
\begin{tabular}[t]{|c|c|c|c|c|c|c|c|c|c|}\hline
学籍番号 & & & & & & & & \\ \hline
\end{tabular}\\
学籍番号の\underline{数字の}右から一番目が{\bfseries 偶数の人は右側}の問題を
解いて下さい。
\vspace{-5ex}





\subsection*{問1}

以下の配列{\ttfamily b}の要素を全て表示したい。
{\ttfamily while}を使って、そのためのプログラムの続きを書け。
変数{\ttfamily j}を使ってよい。


\ifnum \anss=1
\begin{verbatim}
int j;
double b[5] = {0.11, 0.56, 0.34, 0.77, 0.39};







\end{verbatim}
\else
\begin{verbatim}
int j;
double b[5] = {0.11, 0.56, 0.34, 0.77, 0.39};
j = 0;
while(j < 5){
  printf("%f ", b[j]);
  j = j + 1;
}
\end{verbatim}
\fi
\vspace{5cm}



\subsection*{問2}



二重ループを使って
$1\times1$から$9\times9$までのかけ算(九九)の表を
表示するプログラムを完成させよ。


\ifnum \anss=1
\begin{verbatim}



  while(a <= 9){



    while(b <= 9){




    }




  }
\end{verbatim}
\vspace{1cm}
\else
\begin{verbatim}
  int a = 1, b;
  while(a <= 9){
    b = 1;
    while(b <= 9){
      printf("%d ", a*b);
      b = b + 1;
    }
    a = a + 1;
    printf("\n");
  }
\end{verbatim}
\vspace{5cm}
\fi




%
%以下の配列{\ttfamily b}の、40未満の要素を全て表示したい。
%{\ttfamily while}を使って、そのためのプログラムを書け。
%変数{\ttfamily j}を使ってよい。
%
%
%\ifnum \anss=1
%\begin{verbatim}
%int j, b[5] = {11, 56, 34, 77, 39};
%
%
%
%
%
%
%
%\end{verbatim}
%\else
%\begin{verbatim}
%int j, b[5] = {11, 56, 34, 77, 39};
%j = 0;
%while(j < 5){
%  if(b[j] < 40){
%    printf("%c ", b[j]);
%  }
%  j = j + 1;
%}
%\end{verbatim}
%\fi
%\vspace{5cm}
%










%\vfill




\end{multicols*}


\end{document}
